% !TEX encoding = UTF-8
% !TEX TS-program = pdflatex
% !TEX root = ../tesi.tex

%**************************************************************
\chapter{Verifica e validazione}
\label{cap:testing} % TODO sistemare riferimenti

\section{Analisi statica}
L'analisi statica relizzata grazie all'utilizzo di SonarQube.
La sua esecuzione veniva messa in atto ogni qual volta avveniva una build di Jenkins.
I problemi che SonarQube può segnalare, possono essere di tre tipi:
\begin{itemize}
    \item \bd{bug}: un errore nel codice che richiede di essere corretto il prima immediatamente;
    \item \bd{vulnerabilità}: un punto nel codice che è aperto agli attacchi;
    \item \bd{codesmell}:  un problema di manutenibiltià che rende il codice confuso e difficile da manutenere.
\end{itemize}
Ognuno dei quali può avere un grado di severità differente, ordinate dalla più alla meno importante:
\begin{itemize}
    \item \bd{BLOCKER}
    \item \bd{CRITICAL}
    \item \bd{MAJOR}
    \item \bd{MINOR}
    \item \bd{INFO}.
\end{itemize}

La maggior parte dei problemi segnalati da SonarQube durante il progetto, erano codesmell di diversa severità MINOR o MAJOR.
Per il termine del progetto, la totalità di quelli etichettati come MAJOR sono stati risolti, come richiesto dalle norme aziendali relative al codice, ma anche molti dei MINOR.
 


\section{Test di unità}

L'attività di testing per quel che riguarda i test di unità è stata svolta utilizzando principalmente due framework: JUnit e Mockito.

    \subsection{JUnit}


    \subsection{Mockito}



\section{Test di validazione}


%**************************************************************