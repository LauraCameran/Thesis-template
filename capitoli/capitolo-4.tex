% !TEX encoding = UTF-8
% !TEX TS-program = pdflatex
% !TEX root = ../tesi.tex

%**************************************************************
\chapter{Analisi dei requisiti}
\label{cap:analisi-requisiti}
%**************************************************************

\intro{Tale capitolo ha l’obiettivo di esporre e analizzare i requisiti espliciti e impliciti per la realizzazione del plugin Maven per la pubblicazione di documentazione software.
L'attività di analisi ha funto da base per la fase di progettazione del software, in modo che il prodotto fosse conforme alle richieste dell’azienda.}\\

\section{Casi d'uso}

Per lo studio dei casi di utilizzo del prodotto sono stati creati dei diagrammi.
I diagrammi dei casi d'uso (in inglese \emph{Use Case Diagram}) sono diagrammi di tipo \gls{uml} dedicati alla descrizione delle funzioni o servizi offerti da un sistema, così come sono percepiti e utilizzati dagli attori che interagiscono col sistema stesso.
Essendo il progetto finalizzato alla creazione di un tool per l'automazione di un processo, le interazioni da parte dell'utilizzatore devono essere ovviamente ridotte allo stretto necessario. Per questo motivo i diagrammi d'uso risultano semplici e in numero ridotto.

\begin{figure}[!h] 
    \centering 
    \includegraphics[width=0.9\columnwidth]{usecase/scenario-principale} 
    \caption{Use Case - UC0: Scenario principale}
\end{figure}

\begin{usecase}{0}{Scenario principale}
\usecaseactors{Sviluppatore applicativi}
\usecasepre{Lo sviluppatore è entrato nel plug-in di simulazione all'interno dell'IDE}
\usecasedesc{La finestra di simulazione mette a disposizione i comandi per configurare, registrare o eseguire un test}
\usecasepost{Il sistema è pronto per permettere una nuova interazione}
\label{uc:scenario-principale}
\end{usecase}


\begin{usecase}{1}{Configurazione}
\usecaseactors{Sviluppatore}
\usecasepre{Lo sviluppatore sta utilizzando il progetto}
\usecasedesc{Lo sviluppatore configura come }
\usecasepost{Il sistema è pronto per l'esecuzione}
\label{uc:scenario-principale}
\end{usecase}



\section{Requisiti}
Ad ogni requisito viene assegnato il codice identificativo univoco:
	\begin{center}
		\texttt{R[Numero][Tipo][Priorità]}
	\end{center}
	in cui ogni parte ha un significato preciso:
	\begin{itemize}
		\item \textbf{R}: requisito.
		\item \textbf{Numero}: numero progressivo che segue la struttura dei documenti.
		\item \textbf{Tipo}: la la tipologia di requisito che può essere di:
		\begin{itemize}
			\item \textbf{F}: funzionalità.
			\item \textbf{Q}: qualità.
			\item \textbf{V}: vincolo.
		\end{itemize}
		\item \textbf{Priorità}: indica il grado di urgenza di un requisito di essere soddisfatto, come:
		\begin{itemize}
			\item \textbf{0}: opzionale.
			\item \textbf{1}: desiderabile.
			\item \textbf{2}: obbligatorio.
		\end{itemize}
	\end{itemize}


	Esempio: \texttt{R2Q1} indica il secondo requisito di qualità ed è desiderabile.


\subsection{Tracciamento dei requisiti}

Da un'attenta analisi dei requisiti e dei cais d'uso effettuata sul progetto, è stata stilata la tabella che traccia tutti i requisiti in rapporto alle loro fonti.\\


\newpage

\newcommand{\req}[3]{%
#1 & #2 & #3 \\
}

	%COMANDI PER REQ DI FUNZIONALITÀ
	% Generazione automatica dei numeri
	\newcounter{vaF} % valore
	\newcounter{secF}[vaF] % per il secondo livello del requisito
	\newcounter{thF}[secF] % terzo livello

	\newcommand{\ReqF}[3]{\stepcounter{vaF}R\thevaF F#1 & #2 & #3 \\} % Primo livello
	\newcommand{\subReqF}[3]{\stepcounter{secF}R\thevaF.\thesecF F#1 & #2 & #3 \\} % Secondo livello
	\newcommand{\subsubReqF}[3]{\stepcounter{thF}R\thevaF.\thesecF.\thethF F#1 & #2 & #3 \\} % Terzo livello

	\newcounter{tableCounter} % Per automatizzare conteggio tabelle

	\subsection{Requisiti di funzionalità}\label{RequisitiFunzionalità}

	\stepcounter{tableCounter}
	\begin{table}[H]
		\begin{paddedtablex}[1.7]{\textwidth}{cXc}%0 opz  2 obb
			\rowcolor{beautyblue} \textbf{Codice} & \textbf{Requisito} & \textbf{Fonte} \\\toprule

			% \stepcounter{vaF} % Per allineare i requisiti ai casi d'uso

			% da casi d'uso
			\rowcolor{lightgray}\ReqF{2}{Redmine deve poter inviare una segnalazione al Producer Redmine}{Interno UC1-PR}
                \rowcolor{white}\subReqF{2}{Redmine deve poter inviare la segnalazione di apertura issue al Producer Redmine}{Interno UC1.1-PR}
                \rowcolor{lightgray}\subReqF{2}{Redmine deve poter inviare la segnalazione di modifica issue al Producer Redmine}{Interno UC1.2-PR}
                \rowcolor{white}\subReqF{2}{Redmine deve poter inviare la segnalazione di commento di una issue al Producer Redmine}{Interno UC1.3-PR}
            \rowcolor{lightgray}\ReqF{2}{GitLab deve essere in grado di inviare una segnalazione al Producer GitLab}{Interno UC2-PG}
			     \rowcolor{white}\subReqF{2}{GitLab deve essere in grado di segnalare l'apertura di issue al Producer GitLab}{Interno UC2.1-PG}
			     \rowcolor{lightgray}\subReqF{2}{GitLab deve essere in grado di segnalare la modifica issue al Producer GitLab}{Interno UC2.2-PG}
                 \rowcolor{white}\subReqF{2}{GitLab deve essere in grado di segnalare il commento di una issue al Producer GitLab}{Interno UC2.3-PG}
			     \rowcolor{lightgray}\subReqF{2}{GitLab deve poter segnalare un evento di push al Producer GitLab}{Interno UC2.4-PG}
                 \rowcolor{white}\subReqF{2}{GitLab deve poter segnalare un evento di commento di commit al Producer GitLab}{Interno UC2.5-PG}
			\rowcolor{lightgray}\ReqF{2}{Il Producer Redmine deve essere in grado di inviare un messaggio al Gestore Personale}{Interno UC3-GP}
				\rowcolor{white}\subReqF{2}{Il Producer Redmine deve essere in grado di inviare un messaggio di apertura issue al Gestore Personale}{Interno UC3.1-GP}
				\rowcolor{lightgray}\subReqF{2}{Il Producer Redmine deve essere in grado di inviare un messaggio di modifica issue al Gestore Personale}{Interno UC3.2-GP}
                \rowcolor{white}\subReqF{2}{Il Producer Redmine deve essere in grado di inviare un messaggio di commento issue al Gestore Personale}{Interno UC3.3-GP}
			\bottomrule
		\end{paddedtablex}
		\caption{Elenco dei requisiti di funzionalità (\thetableCounter)}
	\end{table}




	%COMANDI PER REQUISITI DI Qualità
	% Generazione automatica dei numeri
	\newcounter{vaQ} % valore
	\newcounter{secQ}[vaQ]
	\newcounter{thQ}[secQ] % terzo livello

	\newcommand{\ReqQ}[3]{\stepcounter{vaQ}R\thevaQ Q#1 & #2 & #3 \\}
	\newcommand{\subReqQ}[3]{\stepcounter{secQ}R\thevaQ.\thesecQ Q#1 & #2 & #3 \\}
	\newcommand{\subsubReqQ}[3]{\stepcounter{thQ}R\thevaQ.\thesecQ.\thethQ Q#1 & #2 & #3 \\} % Terzo livello




	\setcounter{tableCounter}{1}
	% NB: molti requisiti di qualità sono stati tolti perchè non sono veri requisiti del sistema, riguardano noi e il nostro modo di fare, non il sistema
	\subsection{Requisiti di qualità}\label{RequisitiQualita}

	\begin{table}[H]
		\begin{paddedtablex}[1.7]{\textwidth}{cXc}
			\rowcolor{beautyblue}\textbf{Codice} & \textbf{Requisito} & \textbf{Fonte} \\
			\toprule
			\ReqQ{1}{La copertura dei test deve essere almeno pari al 70\% del codice}{Interno}
			\ReqQ{1}{Ogni commit effettuato deve rispettare la formattazione descritta nella wiki}{Interno}
			\ReqQ{1}{Ogni documento deve attraversare tutte le fasi previste dal suo ciclo di vita}{Interno QPR008}
            % \ReqQ{1}{Ogni modulo deve essere prima progettato e solo successivamente codificato}{Interno}
            \ReqQ{1}{L'intera progettazione del progetto non deve contenere un numero di pattern più alto di quanto segnalato nel pdq}{Interno QPR012}
			\ReqQ{2}{Le norme presenti sulla wiki aziendale devono essere rispettate}{Interno}
			\ReqQ{2}{Deve essere redatto un manuale utente}{Interno}
			\ReqQ{2}{Deve essere redatto un manuale dello sviluppatore}{Capitolato}
            \ReqQ{1}{Deve essere redatta una pagina di utilizzo Maven ..}{Interno}

			\bottomrule
		\end{paddedtablex}
		\caption{Elenco dei requisiti di qualità.}
	\end{table}











% \begin{table}%
% \caption{Tabella del tracciamento dei requisti funzionali}
% \label{tab:requisiti-funzionali}
% \begin{tabularx}{\textwidth}{lXl}
% \hline\hline
% \textbf{Requisito} & \textbf{Descrizione} & \textbf{Use Case}\\
% \hline
% RFN-1     & L'interfaccia permette di configurare il tipo di sonde del test & UC1 \\
% \hline
% \end{tabularx}
% \end{table}%

% \begin{table}%
% \caption{Tabella del tracciamento dei requisiti qualitativi}
% \label{tab:requisiti-qualitativi}
% \begin{tabularx}{\textwidth}{lXl}
% \hline\hline
% \textbf{Requisito} & \textbf{Descrizione} & \textbf{Use Case}\\
% \hline
% RQD-1    & Le prestazioni del simulatore hardware deve garantire la giusta esecuzione dei test e non la generazione di falsi negativi & - \\
% \hline
% \end{tabularx}
% \end{table}%

% \begin{table}%
% \caption{Tabella del tracciamento dei requisiti di vincolo}
% \label{tab:requisiti-vincolo}
% \begin{tabularx}{\textwidth}{lXl}
% \hline\hline
% \textbf{Requisito} & \textbf{Descrizione} & \textbf{Use Case}\\
% \hline
% RVO-1    & La libreria per l'esecuzione dei test automatici deve essere riutilizzabile & - \\
% \hline
% \end{tabularx}
% \end{table}%