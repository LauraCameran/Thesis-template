% !TEX encoding = UTF-8
% !TEX TS-program = pdflatex
% !TEX root = ../tesi.tex

%**************************************************************
\chapter{Analisi dei requisiti}
\label{cap:analisi-requisiti}
%**************************************************************

\intro{Tale capitolo ha l’obiettivo di esporre e analizzare i requisiti espliciti e impliciti per la realizzazione del plugin Maven per la pubblicazione di documentazione software.
L'attività di analisi ha funto da base per la fase di progettazione del software, in modo che il prodotto fosse conforme alle richieste dell’azienda.}

\section{Casi d'uso}

Per lo studio dei casi di utilizzo del prodotto sono stati creati dei diagrammi.
I diagrammi dei casi d'uso (in inglese \emph{Use Case Diagram}) sono diagrammi di tipo \gls{uml} dedicati alla descrizione delle funzioni o servizi offerti da un sistema, così come sono percepiti e utilizzati dagli attori che interagiscono col sistema stesso.
Essendo il progetto finalizzato alla creazione di un tool per l'automazione di un processo, le interazioni da parte dell'utilizzatore devono essere ovviamente ridotte allo stretto necessario. Per questo motivo i diagrammi d'uso risultano semplici e in numero ridotto.

\begin{figure}[!h]
    \centering
    \includegraphics[width=0.9\columnwidth]{usecase/scenario-principale}
    \caption{Use Case - UC0: Scenario principale}
\end{figure}

\begin{usecase}{0}{Scenario principale}
\usecaseactors{Sviluppatore applicativi}
\usecasepre{Lo sviluppatore è entrato nel plug-in di simulazione all'interno dell'IDE}
\usecasedesc{La finestra di simulazione mette a disposizione i comandi per configurare, registrare o eseguire un test}
\usecasepost{Il sistema è pronto per permettere una nuova interazione}
\label{uc:scenario-principale}
\end{usecase}


\begin{usecase}{1}{Configurazione}
\usecaseactors{Sviluppatore}
\usecasepre{Lo sviluppatore sta utilizzando il progetto}
\usecasedesc{Lo sviluppatore configura come }
\usecasepost{Il sistema è pronto per l'esecuzione}
\label{uc:scenario-principale}
\end{usecase}



\section{Requisiti}
Ad ogni requisito viene assegnato il codice identificativo univoco:
	\begin{center}
		\texttt{R[Numero][Tipo][Priorità]}
	\end{center}
	in cui ogni parte ha un significato preciso:
	\begin{itemize}
		\item \textbf{R}: requisito.
		\item \textbf{Numero}: numero progressivo che segue una struttura gerarchica.
		\item \textbf{Tipo}: la la tipologia di requisito che può essere di:
		\begin{itemize}
			\item \textbf{F}: funzionalità.
			\item \textbf{Q}: qualità.
			\item \textbf{V}: vincolo.
		\end{itemize}
		\item \textbf{Priorità}: indica il grado di urgenza di un requisito di essere soddisfatto, come:
		\begin{itemize}
			\item \textbf{0}: opzionale.
			\item \textbf{1}: desiderabile.
			\item \textbf{2}: obbligatorio.
		\end{itemize}
	\end{itemize}


	Esempio: \texttt{R2Q1} indica il secondo requisito di qualità ed è desiderabile.


\subsection{Tracciamento dei requisiti}

Da un'attenta analisi dei requisiti e dei casi d'uso effettuata sul progetto, è stata stilata la tabella che traccia tutti i requisiti in rapporto alle loro fonti.

% \newpage

	\subsubsection{Requisiti di funzionalità}\label{RequisitiFunzionalità}

	% \stepcounter{tableCounter}
	% TODO chiedere di sta roba
	\begin{table}[H]
		\begin{paddedtablex}[1.7]{\textwidth}{cXc}%0 opz  2 obb
			\rowcolor{beautyblue} \textbf{Codice} & \textbf{Requisito} & \textbf{Fonte} \\\toprule
			\ReqF{2}{L'utente deve poter pubblicare la documentazione da lui scelta}{UC\thevaF}
				\subReqF{2}{L'utente deve poter pubblicare un archivio (.zip o .jar)}{UC\thevaF.\thesecF}
				\subReqF{2}{L'utente deve poter pubblicare un file html}{UC\thevaF.\thesecF}
				\subReqF{2}{L'utente deve poter pubblicare una cartella}{UC\thevaF.\thesecF}
            \ReqF{2}{Inserimento credenziali}{UC\thevaF}
			    \subReqF{2}{Inserimento username}{UC\thevaF.\thesecF}
				\subReqF{2}{Inserimento password}{UC\thevaF.\thesecF}
				\subReqF{2}{Inserimento identificativo server}{UC\thevaF.\thesecF}
			\bottomrule
		\end{paddedtablex}
		\caption{Elenco dei requisiti di funzionalità
		% (\thetableCounter)
		}
	\end{table}




	% \setcounter{tableCounter}{1}
	% NB: molti requisiti di qualità sono stati tolti perchè non sono veri requisiti del sistema, riguardano noi e il nostro modo di fare, non il sistema
	\subsubsection{Requisiti di qualità}\label{RequisitiQualita}

	\begin{table}[H]
		\begin{paddedtablex}[1.7]{\textwidth}{cXc}
			\rowcolor{beautyblue}\textbf{Codice} & \textbf{Requisito} & \textbf{Fonte} \\
			\toprule
			\ReqQ{1}{La copertura dei test deve essere almeno pari al 70\% del codice}{Interno}
            % \ReqQ{1}{Ogni modulo deve essere prima progettato e solo successivamente codificato}{Interno}
            % \ReqQ{1}{L'intera progettazione del progetto non deve contenere un numero di pattern più alto di quanto segnalato nel pdq}{Interno QPR012}
			\ReqQ{1}{Le norme presenti sulla wiki aziendale devono essere rispettate}{Interno}
			\subReqQ{1}{Ogni commit effettuato deve rispettare la formattazione descritta nella wiki}{Interno}
			\ReqQ{2}{Deve essere redatto un manuale utente}{Interno}
			\subReqQ{2}{Deve essere redatta una pagina Confluence che descriva come configurare il plugin}{Interno}
			\subReqQ{1}{Deve essere redatta una pagina di utilizzo Maven ``Usage'' che descriva tutti i possibili utilizzi del plugin}{Interno}
			\ReqQ{2}{Deve essere redatto un manuale dello sviluppatore}{Interno}
			\subReqQ{2}{Deve essere redatta una pagina Confluence che descriva la progettazione del plugin tramite diagrammi}{Interno}
			\subReqQ{2}{Deve essere redatta e generata la documentazione Javadoc del plugin}{Interno}
			\bottomrule
		\end{paddedtablex}
		\caption{Elenco dei requisiti di qualità.}
	\end{table}


	\subsubsection{Requisiti di vincolo}\label{RequisitiVincolo}

	\begin{table}[H]
		\begin{paddedtablex}[1.7]{\textwidth}{cXc}
			\rowcolor{beautyblue}\textbf{Codice} & \textbf{Requisito} & \textbf{Fonte} \\
			\toprule
			\ReqV{2}{Il plugin deve essere sviluppato nel linguaggio di programmazione Java}{Interno}
			\ReqV{2}{Il plugin deve essere testato tramite JUnit}{Interno}
			\ReqV{2}{Come ambiente di sviluppo è necessario utilizzare Eclipse}{Interno}
			\ReqV{2}{Per la build dei progetti è necessario utilizzare Maven}{Interno}
			\ReqV{2}{Per la pubblicazione di documentazione è necessario utilizzare Confluence}{Interno}
			\ReqV{2}{I requisiti identificati devono essere tracciati su Jira}{Interno}
			\subReqV{2}{Lo stato di ogni requisito presente su Jira deve sempre essere opportunamente aggiornato}{Interno}
			\ReqV{2}{Come strumento di Continuous integration è necessario utilizzare Jenkins}{Interno}
			\ReqV{2}{Per l'analisi statica del codice è necessario utilizzare SonarQube}{Interno}
			\ReqV{2}{Per il controllo di versione del codice è necessario utilizzare Bitbucket}{Interno}
			\ReqV{0}{Utilizzare ObjectAid UML Explorer per la creazione dei diagrammi delle classi}{Interno}
			\bottomrule
		\end{paddedtablex}
		\caption{Elenco dei requisiti di vincolo.}
	\end{table}






% \begin{table}%
% \caption{Tabella del tracciamento dei requisti funzionali}
% \label{tab:requisiti-funzionali}
% \begin{tabularx}{\textwidth}{lXl}
% \hline\hline
% \textbf{Requisito} & \textbf{Descrizione} & \textbf{Use Case}\\
% \hline
% RFN-1     & L'interfaccia permette di configurare il tipo di sonde del test & UC1 \\
% \hline
% \end{tabularx}
% \end{table}%

% \begin{table}%
% \caption{Tabella del tracciamento dei requisiti qualitativi}
% \label{tab:requisiti-qualitativi}
% \begin{tabularx}{\textwidth}{lXl}
% \hline\hline
% \textbf{Requisito} & \textbf{Descrizione} & \textbf{Use Case}\\
% \hline
% RQD-1    & Le prestazioni del simulatore hardware deve garantire la giusta esecuzione dei test e non la generazione di falsi negativi & - \\
% \hline
% \end{tabularx}
% \end{table}%

% \begin{table}%
% \caption{Tabella del tracciamento dei requisiti di vincolo}
% \label{tab:requisiti-vincolo}
% \begin{tabularx}{\textwidth}{lXl}
% \hline\hline
% \textbf{Requisito} & \textbf{Descrizione} & \textbf{Use Case}\\
% \hline
% RVO-1    & La libreria per l'esecuzione dei test automatici deve essere riutilizzabile & - \\
% \hline
% \end{tabularx}
% \end{table}%