% !TEX encoding = UTF-8
% !TEX TS-program = pdflatex
% !TEX root = ../tesi.tex

%**************************************************************
\chapter{Analisi dei requisiti}
\label{cap:analisi-requisiti}
%**************************************************************

\intro{Tale capitolo ha l’obiettivo di esporre e analizzare i requisiti espliciti e impliciti per la realizzazione del plugin Maven per la pubblicazione di documentazione software.
L'attività di analisi ha funto da base per la fase di progettazione del software, in modo che il prodotto fosse conforme alle richieste dell’azienda.}

% \section{Casi d'uso}

% Per lo studio dei casi di utilizzo del prodotto sono stati creati dei diagrammi.
% I diagrammi dei casi d'uso (in inglese \emph{Use Case Diagram}) sono diagrammi di tipo \gls{uml} dedicati alla descrizione delle funzioni o servizi offerti da un sistema, così come sono percepiti e utilizzati dagli attori che interagiscono col sistema stesso.
% Essendo il progetto finalizzato alla creazione di un tool per l'automazione di un processo, le interazioni da parte dell'utilizzatore devono essere ovviamente ridotte allo stretto necessario. Per questo motivo i diagrammi d'uso risultano semplici e in numero ridotto.

% \begin{figure}[!h]
%     \centering
%     \includegraphics[width=0.9\columnwidth]{usecase/scenario-principale}
%     \caption{Use Case - UC0: Scenario principale}
% \end{figure}

% \begin{usecase}{0}{Scenario principale}
% \usecaseactors{Sviluppatore applicativi}
% \usecasepre{Lo sviluppatore è entrato nel plug-in di simulazione all'interno dell'IDE}
% \usecasedesc{La finestra di simulazione mette a disposizione i comandi per configurare, registrare o eseguire un test}
% \usecasepost{Il sistema è pronto per permettere una nuova interazione}
% \label{uc:scenario-principale}
% \end{usecase}


% \begin{usecase}{1}{Configurazione}
% \usecaseactors{Sviluppatore}
% \usecasepre{Lo sviluppatore sta utilizzando il progetto}
% \usecasedesc{Lo sviluppatore configura come }
% \usecasepost{Il sistema è pronto per l'esecuzione}
% \label{uc:scenario-principale}
% \end{usecase}

\section{Premessa}
Il prodotto realizzato è un plugin Maven e possiede come nome ufficiale: \emph{Maven documentation publisher plug-in}.
Un plugin Maven è un programma non autonomo che interagisce con la tecnologia Maven per ampliarne o estenderne le funzionalità originarie.

Essendo un plugin Maven, esso deve avere un \emph{goal}, ovvero uno scopo ben preciso.
L'obiettivo all'inizio dello stage era il semplice caricamento di documentazione archiviata su \emph{Docs} di Confluence, per questo motivo è stato scelto di creare il goal denominato \emph{publish}.
Successivamente, nel corso dello stage, sono state aggiunte delle nuove funzionalità e un nuovo goal, dato che le tempistiche pianificate erano ottimistiche e hanno permesso sufficiente tempo per ampliare il prodotto.


\section{Descrizione del prodotto}

\emph{Maven documentation publisher plug-in} supporta la pubblicazione di documentazione in formato HTML. Ha due \emph{goal}:
\begin{itemize}
	\item \bd{publish}: che pubblica la documentazione;
	\item \bd{cleanup}: che elimina la documentazione contentente SNAPSHOT nel nome.
\end{itemize}

Il principale è \emph{publish} e si occupa della pubblicazione su  \emph{Docs} Confluence della documentazione del codice di un qualunque progetto Maven su cui è configurato il plugin.
Questo è possibile perché il plugin \emph{Docs} di Confluence accetta archivi, ovvero file in formato .zip o .jar.
La documentazione in questo formato può essere per esempio la documentazione Javadoc (documentazione del codice sorgente scritto in linguaggio Java) o Open API (specifica per file di interfaccia leggibili dalle macchine per descrivere servizi web RESTful, conosciuta anche come specifica Swagger).
Entrambe Javadoc e Open API sono il tipo di documentazione di maggior interesse per l'azienda da pubblicare sul sistema aziendale. \\
Ogni archivio caricato contribuisce alla creazione di una pagina \emph{doc}.
Ogni \emph{doc} viene identificato univocamente all'interno di una categoria, per questo motivo, il titolo deve essere unico.
Una pagina viene creata se il titolo della documentazione è nuovo, altrimenti la pagina già esistente viene semplicemente aggiornata. \\ 
 
\emph{Maven documentation publisher plug-in} è altamente configurabile, in modo da soddisfare qualunque esigenza dello sviluppatore.
Innanzitutto esso consente all'utente di inserire:
\begin{itemize}
	\item la documentazione;
	\item le proprie credenziali per accedere a Confluence;	
	\item il nome della categoria in cui allocare la documentazione.
\end{itemize}
La documentazione che fornisce l'utente può essere di tre tipi:
\begin{enumerate}
	\item archivio (.zip o .jar);
	\item cartella (contentente più file HTML);
	\item singolo file HTML.
\end{enumerate}
Nei casi 2. e 3. il plugin si occupa anche dell'archiviazione di quei file. \\

I possibili modi per fornire le credenziali sono molteplici:
\begin{itemize}
	\item username e password vengono date direttamente nella configurazione
\end{itemize}
L'utente può inoltre 



username and password are users credentials directly supplied by user;
serverId corresponds to the server id presents in .m2/settings.xml and permits to get credentials there;
In the above example, there are two ways to get credentials. Whether both of them have values, serverId is the used one.

There must be at least one way to get credentials.







Il secondo \emph{goal}, \emph{cleanup}, è nato dalla necessità di eliminare la documentazione relativa ad un prodotto che non è stato rilasciato.
Questo tipo di prodotti presentano ``SNAPSHOT'' nella versione e per questo motivo, anche il titolo della pagina \emph{doc} lo contiene.
Si ha quindi qui a che vedere con la pulizia totale dal plugin \emph{Docs} di tutte queste pagine.

% Main goal is publish and it is bound to package life cycle phase. 

% Furthermore a doc page can be overall identified by its docKey, which is a sequence of category id and doc id.

% User can also configure it to be skipped always or in specific situations and choose doc page title: name and version.



\section{Requisiti}
Ad ogni requisito viene assegnato il codice identificativo univoco:
	\begin{center}
		\texttt{R[Numero][Tipo][Priorità]}
	\end{center}
	in cui ogni parte ha un significato preciso:
	\begin{itemize}
		\item \textbf{R}: requisito.
		\item \textbf{Numero}: numero progressivo che segue una struttura gerarchica.
		\item \textbf{Tipo}: la la tipologia di requisito che può essere di:
		\begin{itemize}
			\item \textbf{F}: funzionalità.
			\item \textbf{Q}: qualità.
			\item \textbf{V}: vincolo.
		\end{itemize}
		\item \textbf{Priorità}: indica il grado di urgenza di un requisito di essere soddisfatto, come:
		\begin{itemize}
			\item \textbf{0}: opzionale.
			\item \textbf{1}: desiderabile.
			\item \textbf{2}: obbligatorio.
		\end{itemize}
	\end{itemize}


	Esempio: \texttt{R2Q1} indica il secondo requisito di qualità ed è desiderabile.



% Da un'attenta analisi dei requisiti e dei casi d'uso effettuata sul progetto, è stata stilata la tabella che traccia tutti i requisiti in rapporto alle loro fonti.

% \newpage

	% TODO dividere i requisiti in altri gruppi e dare una spiegazione
	\subsection{Requisiti di funzionalità}\label{RequisitiFunzionalità}

	% \stepcounter{tableCounter}
	\begin{table}[H]
		\begin{paddedtablex}[1.7]{\textwidth}{cXc}%0 opz  2 obb
			\rowcolor{beautyblue} \textbf{Codice} & \textbf{Requisito} & \textbf{Fonte} \\\toprule
			\ReqF{2}{Il sistema deve fornire delle proprietà per tutti gli elementi configurabili dall'utente}{Azienda}
			\ReqF{2}{L'utente deve poter pubblicare la documentazione da lui scelta}{Azienda}
				\subReqF{2}{L'utente deve poter pubblicare un archivio (.zip o .jar)}{Azienda}
				\subReqF{2}{L'utente deve poter pubblicare un file html}{Azienda}
				\subReqF{2}{L'utente deve poter pubblicare una cartella}{Azienda}	
			\ReqF{2}{Il sistema deve dare un messaggio di errore se l'utente non fornisce nessuna documentazione}{Azienda}
			\ReqF{2}{Il sistema deve dare un messaggio di errore se l'archivio dato non esiste}{Azienda}				
            \ReqF{2}{Inserimento credenziali}{Azienda}
			    \subReqF{2}{Inserimento username}{Azienda}
				\subReqF{2}{Inserimento password}{Azienda}
				\subReqF{2}{Inserimento identificativo server}{Azienda}
			\ReqF{2}{Il sistema deve dare un messaggio di errore se l'utente non fornisce, in almeno uno dei due modi, le sue credenziali}{Azienda}
			\ReqF{2}{L'utente deve poter inserire il nome della categoria Confluence in cui allocare la documentazione}{Azienda}	
			\ReqF{2}{Il sistema deve dare un messaggio di errore se la categoria non è stata aggiunta}{Azienda}
			\ReqF{2}{L'utente deve poter modificare il luogo in cui l'archivio viene salvato all'interno del progetto}{Azienda}
			\ReqF{2}{L'utente deve poter modificare le tipologie di file da inserire nella documentazione}{Azienda}
				\ReqF{2}{L'utente deve poter modificare le tipologie di file da includere nella cartella}{Azienda}
				\ReqF{2}{L'utente deve poter modificare le tipologie di file da escludere dalla cartella}{Azienda}
			\ReqF{2}{Il sistema deve fornire il nome del ``main entrance file'' di ogni pagina Doc del plugin Confluence}{Azienda}
			\ReqF{2}{L'utente deve poter modificare il nome del file principale della documentazione}{Azienda}
			\ReqF{2}{L'utente deve poter inserire il nome della documentazione}{Azienda}
			\bottomrule
		\end{paddedtablex}
		\caption{Elenco dei requisiti di funzionalità (1)}
	\end{table}


	\begin{table}[H]
		\begin{paddedtablex}[1.7]{\textwidth}{cXc}%0 opz  2 obb
			\rowcolor{beautyblue} \textbf{Codice} & \textbf{Requisito} & \textbf{Fonte} \\\toprule
			\ReqF{2}{L'utente deve poter inserire la versione della documentazione}{Azienda}
			\ReqF{1}{Il sistema deve essere in grado di costruire il titolo della pagina contenente la documentazione, a partire da nome e versione della documentazione}{Interno}
			\ReqF{2}{L'utente deve poter configurare il plugin in modo che esso non fallisca se avvengono errori del client}{Azienda}
			\ReqF{2}{L'utente deve poter configurare il plugin in modo che esso ne salti la propria esecuzione}{Azienda}
			\ReqF{2}{L'utente deve poter inserire i tipi di progetto supportati dal plugin}{Azienda}
			\ReqF{2}{Il sistema deve permettere il salto dell'esecuzione del plugin, nel caso in cui il progetto compilato non sia tra i tipi supportati}{Azienda}
			\ReqF{2}{L'utente deve poter inserie i tipi di progetto a cui il plugin non deve dare messaggi di avvertimento}{Azienda}
			\ReqF{0}{L'utente deve poter configurare il plugin in modo che esso non fallisca se l'archivio dato non esiste}{Azienda}
			\ReqF{0}{L'utente deve poter eliminare tutta la documentazione con versione ``SNAPSHOT'' caricata}{Azienda}
			\bottomrule
		\end{paddedtablex}
		\caption{Elenco dei requisiti di funzionalità (2)
		% (\thetableCounter)
		}
	\end{table}



	% \setcounter{tableCounter}{1}
	% NB: molti requisiti di qualità sono stati tolti perchè non sono veri requisiti del sistema, riguardano noi e il nostro modo di fare, non il sistema
	\subsection{Requisiti di qualità}\label{RequisitiQualita}

	\begin{table}[H]
		\begin{paddedtablex}[1.7]{\textwidth}{cXc}
			\rowcolor{beautyblue}\textbf{Codice} & \textbf{Requisito} & \textbf{Fonte} \\
			\toprule
			\ReqQ{1}{La copertura dei test deve essere almeno pari al 70\% del codice}{Azienda}
			\ReqQ{1}{Le norme presenti sulla wiki aziendale devono essere rispettate}{Azienda}
				\subReqQ{1}{Ogni commit effettuato deve rispettare la formattazione descritta nella wiki}{Azienda}
				\subReqQ{1}{Il nome di ogni variabile, classe, ecc nel codice deve essere significativo}{Azienda}
				\subReqQ{1}{I commenti nel codice devono essere facilmente comprensibili}{Azienda}
				\subReqQ{1}{Il codice non deve contenere violazioni di SonarQube con alta severità}{Azienda}
			\ReqQ{1}{Ogni messaggio di errore del plugin deve essere sufficientemente esplicativo}{Interno}
			\bottomrule
		\end{paddedtablex}
		\caption{Elenco dei requisiti di qualità (1)}
	\end{table}


	\begin{table}[H]
		\begin{paddedtablex}[1.7]{\textwidth}{cXc}
			\rowcolor{beautyblue}\textbf{Codice} & \textbf{Requisito} & \textbf{Fonte} \\
			\toprule
			\ReqQ{2}{Deve essere redatto un manuale utente}{Azienda}
				\subReqQ{2}{Deve essere redatta una pagina Confluence che descriva come configurare il plugin}{Azienda}
				\subReqQ{1}{Deve essere redatta una pagina di utilizzo Maven ``Usage'' che descriva tutti i possibili utilizzi del plugin}{Azienda}
			\ReqQ{2}{Deve essere redatto un manuale dello sviluppatore}{Azienda}
				\subReqQ{2}{Deve essere redatta una pagina Confluence che descriva la progettazione del plugin tramite diagrammi}{Azienda}
				\subReqQ{2}{Deve essere redatta e generata la documentazione Javadoc del plugin}{Azienda}
			\bottomrule
		\end{paddedtablex}
		\caption{Elenco dei requisiti di qualità (2)}
	\end{table}





	\subsection{Requisiti di vincolo}\label{RequisitiVincolo}

	\begin{table}[H]
		\begin{paddedtablex}[1.7]{\textwidth}{cXc}
			\rowcolor{beautyblue}\textbf{Codice} & \textbf{Requisito} & \textbf{Fonte} \\
			\toprule
			\ReqV{2}{Il plugin deve essere sviluppato nel linguaggio di programmazione Java}{Azienda}
			\ReqV{2}{Il plugin deve essere testato tramite JUnit}{Azienda}
			\ReqV{2}{Come ambiente di sviluppo è necessario utilizzare Eclipse}{Azienda}
			\ReqV{2}{Per la build dei progetti è necessario utilizzare Maven}{Azienda}
			\ReqV{2}{Per la pubblicazione di documentazione è necessario utilizzare Confluence}{Azienda}
			\ReqV{2}{I requisiti identificati devono essere tracciati su Jira}{Azienda}
				\subReqV{2}{Lo stato di ogni requisito presente su Jira deve sempre essere opportunamente aggiornato}{Azienda}
			\ReqV{2}{Come strumento di Continuous integration è necessario utilizzare Jenkins}{Azienda}
			\ReqV{2}{Per l'analisi statica del codice è necessario utilizzare SonarQube}{Azienda}
			\ReqV{2}{Per il controllo di versione del codice è necessario utilizzare Bitbucket}{Azienda}
			\bottomrule
		\end{paddedtablex}
		\caption{Elenco dei requisiti di vincolo (1)}
	\end{table}


	\begin{table}[H]
		% \centering
		% {\def\arraystretch{1.7}
		% \begin{tabularx}{\textwidth}{cXc}
		\begin{paddedtablex}[1.7]{\textwidth}{cXc}
			\rowcolor{beautyblue}\textbf{Codice} & \textbf{Requisito} & \textbf{Fonte} \\
			\toprule
			\ReqV{0}{Utilizzare GitKraken come client di Git}{Interno}
			\ReqV{0}{Utilizzare JUnit per realizzare test di unità}{Interno}
			\ReqV{0}{Utilizzare Visual Studio Code come editor per il codice}{Interno}
			\ReqV{0}{Utilizzare SequenceDiagram.org per la creazione dei diagrammi di sequenza}{Interno}
			\ReqV{0}{Utilizzare ObjectAid UML Explorer per la creazione dei diagrammi delle classi}{Interno}
			\ReqV{0}{Utilizzare Meecrowave per la creazione di un semplice server}{Azienda}
			\bottomrule
		\end{paddedtablex}
		\caption{Elenco dei requisiti di vincolo (2)}
	\end{table}


	\subsection{Riepilogo dei requisiti}

	\begin{table}[H]
		\centering
		{\def\arraystretch{1.7}
		\begin{tabularx}{\textwidth}{XXXX}
			\rowcolor{beautyblue} \textbf{Tipologia} & \textbf{Obbligatori} & \textbf{Desiderabili} & \textbf{Opzionali} \\\toprule
			Di funzionalità & 28 & 1 & 0 \\
			Di qualità & 5 & 8 & 0 \\
			Di vincolo & 10 & 0 & 6
			\\\bottomrule
		\end{tabularx}}
		\caption{Riepilogo dei requisiti}
	\end{table}





% \begin{table}%
% \caption{Tabella del tracciamento dei requisiti di vincolo}
% \label{tab:requisiti-vincolo}
% \begin{tabularx}{\textwidth}{lXl}
% \hline\hline
% \textbf{Requisito} & \textbf{Descrizione} & \textbf{Use Case}\\
% \hline
% RVO-1    & La libreria per l'esecuzione dei test automatici deve essere riutilizzabile & - \\
% \hline
% \end{tabularx}
% \end{table}%