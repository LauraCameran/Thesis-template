% !TEX encoding = UTF-8
% !TEX TS-program = pdflatex
% !TEX root = ../tesi.tex

%**************************************************************
\chapter{Introduzione}
\label{cap:introduzione}
%**************************************************************

% Introduzione al contesto applicativo.\\

% \noindent Esempio di utilizzo di un termine nel glossario \\
% \gls{api}. \\

% \noindent Esempio di citazione in linea \\
% \cite{site:agile-manifesto}. \\

% \noindent Esempio di citazione nel pie' di pagina \\
% citazione\footcite{womak:lean-thinking} \\

%**************************************************************
\section{Il progetto}
%Descrizione dell'azienda.
Finantix è un'azienda di informatica che vende prodotti software.
Il suo prodotto principale è suddiviso in moduli.
Ognuno di questi moduli prevede una propria documentazione delle API Java (un archivio zip contente documentazione in formato Javadoc) e la documentazione della API RESTful (un archivio zip contenente documentazione in formato Open API).
Questa documentazione viene manualmente caricata sulla piattaforma Confluence ove cui è consultata. 

%Introduzione all'idea dello stage.
Il plugin Maven nasce dalla necessità di automatizzare la pubblicazione di questa documentazione su Confluence, in modo da semplificare e velocizzare notevolmente questo processo.
Infatti, una volta configurato correttamente il plugin in tutti i progetti relativi ai moduli software, il caricamento avviene direttamente durante la build dei progetti, senza richiedere ulteriore intervento umano.

%**************************************************************
\section{Principali problematiche}
Le principali problematiche riscontrate 

%**************************************************************
\subsection{Soluzioni scelte}
.....

%**************************************************************
\section{Strumenti utilizzati}
Qui di seguito viene riportata una tabelle con tutte le tecnologie utilizzate e a quale scopo.

\begin{table}[H]
	{\def\arraystretch{1.5}
    \begin{tabularx}{\textwidth}{c | X}
        % \rowcolor{gray!30}
        \textbf{Tecnologie} &
        \textbf{Scopo} \\ \hline
        Eclipse & Ambiente di sviluppo \\
        Maven & Build automation per la gestione di progetti \\
        Confluence & Pubblicazione, creazione e consultazione di documentazione \\
        Jira & Issue tracking system \\
        Jenkins & Continuous integration \\
        Sonarqube & Analisi statica del codice \\
        Bitbucket e GitKraken & Controllo di versione \\
        JUnit & Test di unità \\
        Visual Studio Code & Editor di codice \\
        SequenceDiagram.org & Creazione dei diagrammi di sequenza \\
        ObjectAid UML Explorer & Creazione dei diagrammi delle classi \\
        Meecrowave & Creazione di server \\
    \end{tabularx}} \\
\caption{Tabella di tecnologie utilizzate durante il progetto e loro scopo.}
\end{table}


%**************************************************************
\section{Il prodotto ottenuto}
.....

%**************************************************************
\section{Organizzazione del testo}

\begin{description}
    \item[{\hyperref[cap:analisi-requisiti]{Il secondo capitolo}}] comprende l'analisi dettagliata dei requisiti del prodotto con casi d'uso e il relativo tracciamento dei requisiti individuati.
    
    \item[{\hyperref[cap:progettazione]{Il terzo capitolo}}] descrive la progettazione del software.
    
    \item[{\hyperref[cap:realizzazione-testing]{Il quarto capitolo}}] approfondisce la realizzazione del plugin e come è stata effettuata l'attività di testing.
    
    \item[{\hyperref[cap:conclusioni]{Il quinto capitolo}}] corrisponde al capitolo conclusivo. Esso riassume il risultato finale ottenuto e attua una valutazine critica del prodotto.
    
\end{description}

Riguardo la stesura del testo, relativamente al documento sono state adottate le seguenti convenzioni tipografiche:
\begin{itemize}
	\item gli acronimi, le abbreviazioni e i termini ambigui o di uso non comune menzionati vengono definiti nel glossario, situato alla fine del presente documento;
	\item per la prima occorrenza dei termini riportati nel glossario viene utilizzata la seguente nomenclatura: \emph{parola}\glsfirstoccur;
	\item i termini in lingua straniera o facenti parti del gergo tecnico sono evidenziati con il carattere \emph{corsivo}.
\end{itemize}